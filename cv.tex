%!TEX TS-program = xelatex
\documentclass[]{friggeri-cv}

\usepackage{enumitem} \renewenvironment{entrylist}{ \begin{itemize}[leftmargin=1in]}{ \end{itemize} } \renewcommand{\bfseries}{\headingfont\color{headercolor}} \renewcommand{\entry}[4]{ \item[#1] \textbf{#2} \hfill {\footnotesize\addfontfeature{Color=lightgray} #3}\\ #4\vspace{\parsep} }

\begin{document}
\header{Moritz Schäfer}{}
       {computer scientist}


% In the aside, each new line forces a line break
\begin{aside}
  \section{about}
    Hessische Straße 12
    10115 Berlin
    Germany
    ~
    \href{mailto:mail@moritz.de}{mail@moritzs.de}
    \href{https://github.com/moritzschaefer}{github.com/moritzschaefer}
  \section{languages}
    german
    english (toefl C1)
    spanish (fluent)
    chinese (HSK 3/B1)
    french notions
  \section{coding}
    %{\color{red} $\varheartsuit$} JavaScript
    linux, docker, GPGPU, ML/data analysis, databases
    Web:
    python
    HTML5(JS \& CSS3); frameworks: ReactJS, AngularJS, Flask
    former:
    C/C++, 3D/OpenGL, Android
\end{aside}


\section{interests}

computational biology, genomics, protein folding, \\
machine learning, data mining and analytics, \\
web technologies%(full stack development),
% hardware/electronics(pi/arduino), bluetooth le,

\section{education}

\begin{entrylist}
  \entry
    {2017-2018}
    {master thesis}
    {Bayer AG @ oncology target discovery}
    {Bioinformatics}
  \entry
    {2016-2018}
    {M.Sc.}
    {TU Berlin}
    {Computer Science}
  \entry
    {2015–2016}
    {bachelor thesis (1.3) \& M.Sc.}
    {Jiao Tong University, Shanghai}
    {Computer Science}
  \entry
    {2013–2014}
    {B.Sc.\ exchange program}
    {Universidad de La Laguna, Tenerife (Spain)}
    {Computer Science (Erasmus)}
  \entry
    {2012–2016}
    {B.Sc.\ (1.9)}
    {Technische Universität, Berlin}
    {Computer Science}
  \entry
    {2011}
    {German Abitur (1.2) with honors}
    {Gewerbliche Schule, Tübingen}
    {Specialization in computer science}
\end{entrylist}

\section{experience}

\begin{entrylist}
  \entry
    {2017}
    {{Antavi GmbH}}
    {Software developer}
    {\emph{Android \& iOS development}}
  \entry
    {2016}
    {{Lamoda Group}}
    {DevOps and software engineer}
    {\emph{Marketing analysis software}}
  \entry
    {2015}
    {\href{http://www.lamoda.ru/}{Lamoda}}
    {Software developer}
    {\emph{Marketing analysis software}}
  \entry
    {2014-2015}
    {\href{http://www.rocket-internet.com/}{Rocket Internet}}
    {Software developer}
    {\emph{Big Data analysis and optimization for marketing department}}
  \entry
    {2013-2014}
    {\href{http://www.lamudi.com/}{Lamudi/Carmudi}}
    {Software developer}
    {\emph{data mining for marketing department}}
  \entry
    {2012-2014}
    {\href{http://brandpunkt.com}{BRANDPUNKT}}
    {Student Job}
    {\emph{Facebook HTML5 Desktop and Mobile Apps}}
  \entry
    {2011/2012}
    {Sayarinapaj}
    {Volunteering}
    {\emph{One year volunteering in Bolivia}}
  \entry
    {2010/2011}
    {Freelancing}
    {Software developer}
    {\emph{Creating dynamic on-demand web solutions}}
  \entry
    {2009-2011}
    {\href{http://www.somfy.de}{Somfy.de}}
    {Student job}
    {\emph{Developing and hardware administration for marketing department }}
  \entry
    {2009 \& 2010}
    {\href{http://jugend-forscht.de}{Jugend forscht e.V.}}
    {research competition}
    {\emph{ultrasonic range analyzer for 3d locating(2009), GPGPU ray tracing(2010)}}
\end{entrylist}

% \section{applications}
%
% \begin{entrylist}
%   \entry
%     {yearn}
%     {title/name}
%     {\href{url}{url title}}
%     {explanation}
% \end{entrylist}

\section{publications}

\begin{entrylist}
  \entry
    {2017}
    {Neural Patterns between Chinese and Germans for EEG-based Emotion Recognition}
    {\href{http://ieeexplore.ieee.org/abstract/document/8008300/?reload=true}{PDF}}
    {}
  \entry
    {2015}
    {A reliable multi-channel RSS measurement tool for Wireless Sensor Networks}
    {\href{http://www.tkn.tu-berlin.de/index.php?id=159554}{TKN Project website}}
    {}
\end{entrylist}
\end{document}
